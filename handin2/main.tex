\documentclass[a4paper]{article}

% --- Packages ---

\usepackage{a4wide}
\usepackage[utf8]{inputenc}
\usepackage{amsmath}
\usepackage{mathtools}
\usepackage{amssymb}
\usepackage[english]{babel}
\usepackage{mdframed}
\usepackage{systeme,}
\usepackage{lipsum}
\usepackage{relsize}
\usepackage{caption}
\usepackage{tikz}
\usepackage{tikz-3dplot}
\usetikzlibrary{shapes.geometric}
\usepackage{pgfplots}
\usepackage{pgfplotstable}
\pgfplotsset{compat=newest}%1.7}
\usepackage{harpoon}%
\usepackage{graphicx}
\usepackage{wrapfig}
\usepackage{subcaption}
\usepackage{authblk}
\usepackage{float}
\usepackage{listings}
\usepackage{xcolor}
\usepackage{chngcntr}
\usepackage{amsthm}
\usepackage{comment}
\usepackage{commath}
\usepackage{hyperref}%Might remove, adds link to each reference
\usepackage{url}
\usepackage{calligra}

% --- Commands --- 

\newcommand{\w}{\omega}
\newcommand{\trace}{\text{Tr}}
\newcommand{\grad}{\mathbf{\nabla}}
%\newcommand{\crr}{\mathfrak{r}}
\newcommand{\laplace}{\nabla^2}
\newcommand{\newparagraph}{\vspace{.5cm}\noindent}

% --- Math character commands ---

\newcommand{\curl}[1]{\mathbf{\nabla}\times \mathbf{#1}}
\newcommand{\dive}[1]{\mathbf{\nabla}\cdot \mathbf{#1}}
\newcommand{\res}[2]{\text{Res}(#1,#2)}
\newcommand{\fpartial}[2]{\frac{\partial #1}{\partial #2}}
\newcommand{\rot}[3]{\begin{vmatrix}\hat{x}&\hat{y}&\hat{z}\\\partial_x&\partial_y&\partial_z\\#1&#2&#3 \end{vmatrix}}
\newcommand{\average}[1]{\langle #1 \rangle}
\newcommand{\ket}[1]{|#1\rangle}
\newcommand{\bra}[1]{\langle #1|}


%  --- Special character commands ---

\DeclareMathAlphabet{\mathcalligra}{T1}{calligra}{m}{n}
\DeclareFontShape{T1}{calligra}{m}{n}{<->s*[2.2]callig15}{}
\newcommand{\crr}{\mathcalligra{r}\,}
\newcommand{\boldscriptr}{\pmb{\mathcalligra{r}}\,}


\title{FK7058: Handin2}
\author{Author : Andreas Evensen}
\date{Date: \today}

% --- Code ---

\definecolor{codegreen}{rgb}{0,0.6,0}
\definecolor{codegray}{rgb}{0.5,0.5,0.5}
\definecolor{codepurple}{rgb}{0.58,0,0.82}
\definecolor{backcolour}{rgb}{0.95,0.95,0.92}

\lstdefinestyle{mystyle}{
    backgroundcolor=\color{backcolour},   
    commentstyle=\color{codegreen},
    keywordstyle=\color{magenta},
    numberstyle=\tiny\color{codegray},
    stringstyle=\color{codepurple},
    basicstyle=\ttfamily\footnotesize,
    breakatwhitespace=false,         
    breaklines=true,                 
    captionpos=b,                    
    keepspaces=true,                 
    numbers=left,                    
    numbersep=5pt,                  
    showspaces=false,                
    showstringspaces=false,
    showtabs=false,                  
    tabsize=2
}

\lstset{style=mystyle}

\begin{document}

\maketitle
\noindent
Consider Ising model, with the following Hamiltonian:
\begin{align*}
    -\mathcal{H} &= H(S_1 + S_2) + JS_1S_2,
\end{align*}where $J >0$ and $S_i = \pm 1 \forall i$.

\subsection*{1)}
Enumerate all the microstates of the system as well as the probabilities of these microstates.

\newparagraph
\textbf{Answer:} All the possible states are the combination of the two spins $S_1$ and $S_2$.
\begin{enumerate}
    \item $S_1 = 1$ and $S_2 = 1$. 
    \item $S_1 = 1$ and $S_2 = -1$. 
    \item $S_1 = -1$ and $S_2 = 1$. 
    \item $S_1 = -1$ and $S_2 = -1$. 
\end{enumerate}There is an equal probability for each state and since there are four microstates, the probability of each state is $1/4$.

\subsection*{2)}
Calculate the partition function.

\newparagraph
\textbf{Answer:} The partition function is given as:
\begin{align*}
    \sum_i e^{-\beta E_i} &= e^{\beta(2H + J)} + 2e^{\beta J} + e^{\beta(-2H + J)}\\
    &= e^{\beta J}\left[2 + e^{2\beta H} + e^{-2\beta H}\right]\\
    &= e^{\beta J}\left[2 + 2\cosh(2\beta H)\right]
\end{align*}

\subsection*{3)}
How does the partition function (and hence the free energy) change if we replace $H$ by $-H$?

\newparagraph
\textbf{Answer:}If instead $H$ is replaced by $-H$ the partition function becomes:
\begin{align*}
    Z &= \sum_i e^{-\beta E_i}\\
    &=e^{-\beta E_1} + e^{-\beta E_2} + e^{-\beta E_3} + e^{-\beta E_4}\\
    &= e^{-\beta (-2H + J)} + 2e^{\beta (J)} + e^{-\beta (2H + J)}\\
    &= e^{\beta J}\left[2 + 2\cosh(2\beta H)\right].
\end{align*}The partition function does not change, and thus, neither does the free energy.

\subsection*{4)}
Calculate the average energy of this system.

\newparagraph
\textbf{Answer:} The average energy of the system is given by:
\begin{align*}
    \average{E} &= \frac{1}{Z}\frac{\partial Z}{\partial \beta}\\
    &= -\frac{1}{Z}\frac{\partial}{\partial \beta}\left[e^{\beta J}\left[2 + 2\cosh(2\beta H)\right]\right]\\
    &= -\frac{1}{Z}\left[e^{\beta J}J\left[2 + 2\cosh(2\beta H)\right] + e^{\beta J}\left[4H\sinh(2\beta H)\right]\right]\\
    &= -\frac{1}{Z}e^{\beta J}\left[J\left[2 + 2\cosh(2\beta H)\right] + \left[4H\sinh(2\beta H)\right]\right]\\
    &= -\frac{J\left(2 + 2\cosh(2\beta H)\right) + 4H\sinh(2\beta H)}{2 + 2\cosh(2\beta H)}\\
    &=  -\left[J+ 2H\tanh(\beta H)\right]
\end{align*}

\subsection*{5)}
Calculate the magnetization per unit spin $M =\langle\frac{M_1 + M_2}{2}\rangle$. How is $M$ related to the free energy of the system?
What is $M$ at $H = 0$?

\newparagraph
\textbf{Answer:}

\subsection*{6)}
What is $M$ in the limit $T\to 0$?

\newparagraph
\textbf{Answer:}

\subsection*{7)}
Calculate $\average{S_1 \cdot S_2} - \average{S_1}\average{S_2}$ the connected correlation function.

\newparagraph
\textbf{Answer:} We begin by computing the average of $S_1$ and $S_2$:
\begin{align*}
    \average{S_1} &= \sum_i S_1^{(i)} \cdot p_i = \frac{1}{4} - \frac{1}{4} = 0,\\
    \average{S_2} &= \sum_i S_2^{(i)}\cdot p_i = \frac{1}{4} - \frac{1}{4} = 0.
\end{align*}

\subsection*{8)}
Show that, if we change the Hamiltonian of the system so that:
\begin{align*}
    -\mathcal{H} = H_1S_1 + H_2S_2 + JS_1S_2,
\end{align*}the connected correlation function can be written as the second derivative of the free energy.

\newparagraph
\textbf{Answer:}


\end{document}
 
