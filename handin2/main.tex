\documentclass[a4paper]{article}

% --- Packages ---

\usepackage{a4wide}
\usepackage[utf8]{inputenc}
\usepackage{amsmath}
\usepackage{mathtools}
\usepackage{amssymb}
\usepackage[english]{babel}
\usepackage{mdframed}
\usepackage{systeme,}
\usepackage{lipsum}
\usepackage{relsize}
\usepackage{caption}
\usepackage{tikz}
\usepackage{tikz-3dplot}
\usetikzlibrary{shapes.geometric}
\usepackage{pgfplots}
\usepackage{pgfplotstable}
\pgfplotsset{compat=newest}%1.7}
\usepackage{harpoon}%
\usepackage{graphicx}
\usepackage{wrapfig}
\usepackage{subcaption}
\usepackage{authblk}
\usepackage{float}
\usepackage{listings}
\usepackage{xcolor}
\usepackage{chngcntr}
\usepackage{amsthm}
\usepackage{comment}
\usepackage{commath}
\usepackage{hyperref}%Might remove, adds link to each reference
\usepackage{url}
\usepackage{calligra}

% --- Commands --- 

\newcommand{\w}{\omega}
\newcommand{\trace}{\text{Tr}}
\newcommand{\grad}{\mathbf{\nabla}}
%\newcommand{\crr}{\mathfrak{r}}
\newcommand{\laplace}{\nabla^2}
\newcommand{\newparagraph}{\vspace{.5cm}\noindent}

% --- Math character commands ---

\newcommand{\curl}[1]{\mathbf{\nabla}\times \mathbf{#1}}
\newcommand{\dive}[1]{\mathbf{\nabla}\cdot \mathbf{#1}}
\newcommand{\res}[2]{\text{Res}(#1,#2)}
\newcommand{\fpartial}[2]{\frac{\partial #1}{\partial #2}}
\newcommand{\rot}[3]{\begin{vmatrix}\hat{x}&\hat{y}&\hat{z}\\\partial_x&\partial_y&\partial_z\\#1&#2&#3 \end{vmatrix}}
\newcommand{\average}[1]{\langle #1 \rangle}
\newcommand{\ket}[1]{|#1\rangle}
\newcommand{\bra}[1]{\langle #1|}


%  --- Special character commands ---

\DeclareMathAlphabet{\mathcalligra}{T1}{calligra}{m}{n}
\DeclareFontShape{T1}{calligra}{m}{n}{<->s*[2.2]callig15}{}
\newcommand{\crr}{\mathcalligra{r}\,}
\newcommand{\boldscriptr}{\pmb{\mathcalligra{r}}\,}


\title{FK7058: Handin2}
\author{Author : Andreas Evensen}
\date{Date: \today}

% --- Code ---

\definecolor{codegreen}{rgb}{0,0.6,0}
\definecolor{codegray}{rgb}{0.5,0.5,0.5}
\definecolor{codepurple}{rgb}{0.58,0,0.82}
\definecolor{backcolour}{rgb}{0.95,0.95,0.92}

\lstdefinestyle{mystyle}{
    backgroundcolor=\color{backcolour},   
    commentstyle=\color{codegreen},
    keywordstyle=\color{magenta},
    numberstyle=\tiny\color{codegray},
    stringstyle=\color{codepurple},
    basicstyle=\ttfamily\footnotesize,
    breakatwhitespace=false,         
    breaklines=true,                 
    captionpos=b,                    
    keepspaces=true,                 
    numbers=left,                    
    numbersep=5pt,                  
    showspaces=false,                
    showstringspaces=false,
    showtabs=false,                  
    tabsize=2
}

\lstset{style=mystyle}

\begin{document}

\maketitle
\noindent
Consider Ising model, with the following Hamiltonian:
\begin{align*}
    -\mathcal{H} &= H(S_1 + S_2) + JS_1S_2,
\end{align*}where $J >0$ and $S_i = \pm 1 \forall i$.

\subsection*{1)}
Enumerate all the microstates of the system as well as the probabilities of these microstates.

\newparagraph
\textbf{Answer:} All the possible states are the combination of the two spins $S_1$ and $S_2$.
\begin{enumerate}
    \item $S_1 = 1$ and $S_2 = 1$, $-\mathcal{H} = 2H + J$. 
    \item $S_1 = 1$ and $S_2 = -1$, $-\mathcal{H} = -J$. 
    \item $S_1 = -1$ and $S_2 = 1$, $-\mathcal{H} = -J$. 
    \item $S_1 = -1$ and $S_2 = -1$, $-\mathcal{H} = -2H + J$. 
\end{enumerate}The probability of the states is given by the following:
\begin{align*}
    P(S_1, S_2) &= \frac{e^{-\beta E_i}}{Z},\\
    P(1, 1) &= \frac{e^{\beta(2H + J)}}{Z},\\
    P(1, -1) &= \frac{e^{\beta(-J)}}{Z},\\
    P(-1, 1) &= \frac{e^{\beta(-J)}}{Z},\\
    P(-1, -1) &= \frac{e^{\beta(-2H + J)}}{Z}.
\end{align*}

\subsection*{2)}
Calculate the partition function.

\newparagraph
\textbf{Answer:} The partition function is given as:
\begin{align*}
    Z &= \sum_i e^{-\beta E_i} = e^{\beta(2H + J)} + 2e^{-\beta J} + e^{\beta(-2H + J)}\\
    &= e^{\beta J}\left[2e^{-2\beta J} + e^{2\beta H} + e^{-2\beta H}\right]\\
    &= e^{\beta J}\left[2e^{-2\beta J} + 2\cosh(2\beta H)\right]
\end{align*}

\subsection*{3)}
How does the partition function (and hence the free energy) change if we replace $H$ by $-H$?

\newparagraph
\textbf{Answer:}If instead $H$ is replaced by $-H$ the partition function becomes:
\begin{align*}
    Z &= \sum_i e^{-\beta E_i}\\
    &=e^{-\beta E_1} + e^{-\beta E_2} + e^{-\beta E_3} + e^{-\beta E_4}\\
    &= e^{-\beta (-2H + J)} + 2e^{-\beta (J)} + e^{-\beta (2H + J)}\\
    &= e^{\beta J}\left[2e^{-2\beta J} + 2\cosh(2\beta H)\right].
\end{align*}The partition function does not change, and thus, neither does the free energy.

\subsection*{4)}
Calculate the average energy of this system.

\newparagraph
\textbf{Answer:} The average energy of the system is given by:
\begin{align*}
    \average{E} &= \frac{1}{Z}\frac{\partial Z}{\partial \beta}\\
    &= -\frac{1}{Z}\frac{\partial}{\partial \beta}\left[e^{\beta J}\left[2e^{-2\beta J} + 2\cosh(2\beta H)\right]\right]\\
    &= -\frac{1}{Z}\left[Je^{\beta J}\left[2e^{-2\beta J} + 2\cosh(2\beta H)\right] + e^{\beta J}\left(-4Je^{-2\beta J} +4H\sinh(2\beta H)\right)\right]\\
    &= -\frac{Je^{\beta J}\left[2e^{-2\beta J} + 2\cosh(2\beta H)\right] + e^{\beta J}\left(-4Je^{-2\beta J} +4H\sinh(2\beta H)\right)}{e^{\beta J}\left[2e^{-2\beta J} + 2\cosh(2\beta H)\right]}\\
    &= -J + \frac{4Je^{-2\beta J} - 4H\sinh(2\beta H)}{2\cosh(2\beta H) + 2e^{-2\beta J}}
\end{align*}
\begin{comment}
\begin{align*}
    \average{E} &= \frac{1}{Z}\frac{\partial Z}{\partial \beta}\\
    &= -\frac{1}{Z}\frac{\partial}{\partial \beta}\left[e^{\beta J}\left[2 + 2\cosh(2\beta H)\right]\right]\\
    &= -\frac{1}{Z}\left[e^{\beta J}J\left[2 + 2\cosh(2\beta H)\right] + e^{\beta J}\left[4H\sinh(2\beta H)\right]\right]\\
    &= -\frac{1}{Z}e^{\beta J}\left[J\left[2 + 2\cosh(2\beta H)\right] + \left[4H\sinh(2\beta H)\right]\right]\\
    &= -\frac{J\left(2 + 2\cosh(2\beta H)\right) + 4H\sinh(2\beta H)}{2 + 2\cosh(2\beta H)}\\
    &=  -\left[J+ 2H\tanh(\beta H)\right]
\end{align*}
\end{comment}

\subsection*{5)}
Calculate the magnetization per unit spin $M =\langle\frac{M_1 + M_2}{2}\rangle$. How is $M$ related to the free energy of the system?
What is $M$ at $H = 0$?

\newparagraph
\textbf{Answer:} The magnetization per unit spin $M$ is defined as:
\begin{align*}
    M &= \frac{1}{N}\sum_{i = 1}^N \average{S_i} = \frac{\average{S_1} + \average{S_2}}{2}
\end{align*}
We can now write, that in general, the magnetization per unit spin is given by:
\begin{align*}
    M &= -\frac{1}{N}\frac{\partial\mathcal{F}}{\partial H}\\
    &= -\frac{1}{2}\frac{\partial \mathcal{F}}{\partial H}.
\end{align*}Thus, one has that the magnetization per unit spin is related to the free energy of the system, and to the average sin of the system.
One computes the derivative of the free energy with respect to $H$:
\begin{align*}
    M &=-\frac{1}{2}\frac{\partial F}{\partial H} = -\frac{1}{2}\frac{\partial}{\partial H}\left(-k_bT\ln(Z)\right)\\
    &=\frac{k_bT}{2}\frac{\partial}{\partial H}\left[\ln\left(e^{\beta J}\left[2e^{-2\beta J} + 2\cosh(2\beta H)\right]\right)\right]\\
    &=\frac{k_bT}{2}\frac{\partial}{\partial H}\left(\ln\left[2e^{-2\beta J} + 2\cosh(2\beta H)\right]\right)\\
    &=\frac{1}{2}\frac{2 \sinh(2\beta H)}{e^{-J\beta} + 2\cosh(2\beta H)}.
\end{align*}When $H = 0$ one has that the magnetization per unit spin is given by:
\begin{align*}
    M(H = 0) &= \frac{1}{2}\frac{2 \cdot 0}{e^{-J\beta} + 2} = 0.
\end{align*}
\begin{comment}
One can write the $\average{S_1}$ and $\average{S_2}$ as:
\begin{align*}
    \average{S_1} &= -\frac{\partial\mathcal{F}}{\partial H_1},\\
    \average{S_2} &= -\frac{\partial\mathcal{F}}{\partial H_2},
\end{align*}if one assumes that the Hamiltonian is on the form $-\mathcal{H} = H_1S_1 + H_2S_2 + JS_1S_2$. The free energy is then given by:
\begin{align*}
    \mathcal{F} &= -k_bT\ln(Z),
\end{align*}where the partition function is given by;
\begin{align*}
    Z &= \sum_{S_1}\sum_{S_2}\exp\left[\beta\left(H_1S_1 + H_2S_2 + JS_1S_2\right)\right]\\
    &= \exp\left[\beta H_1 + \beta H_2 + J\right] + \exp\left[\beta H_1 - \beta H_2 - J\right] + \exp\left[-\beta H_1 + \beta H_2 - J\right] + \exp\left[-\beta H_1 - \beta H_2 + J\right]\\
    &= e^{\beta J}\left[e^{\beta(H_1 + H_2)} + e^{-\beta(H_1 + H_2)}\right] + e^{-\beta J}\left[e^{\beta(-H_1 + H_2)} + e^{-\beta(-H_1 + H_2)}\right]\\
    &= e^{\beta J}2\cosh\left(\beta(H_1 + H_2)\right) + e^{-\beta J}2\cosh\left(\beta(-H_1 + H_2)\right).
\end{align*}Thus, the free energy is given by:
\begin{align*}
    \mathcal{F} &= -k_bT\ln\left(e^{\beta J}2\cosh\left(\beta(H_1 + H_2)\right) + e^{-\beta J}2\cosh\left(\beta(-H_1 + H_2)\right)\right).
\end{align*}The averages of $S_1$ and $S_2$ are then given by:
\begin{align*}
    \average{S_1} &= -\frac{\partial\mathcal{F}}{\partial H_1}\\
    &= -\frac{\partial}{\partial H_1}\left[-k_bT\ln\left(e^{\beta J}2\cosh\left(\beta(H_1 + H_2)\right) + e^{-\beta J}2\cosh\left(\beta(-H_1 + H_2)\right)\right)\right]\\
    &= - k_bT\left[2\beta e^{-\beta J }\sinh\left(\beta[H_1 - H_2]\right) + \beta\right]\cdot\tanh\left[2e^{-\beta J}\cosh\left(\beta[H_2 - H_1]\right) + \beta(H_1 + H_2)\right],\\
    \average{S_2} &= -\frac{\partial\mathcal{F}}{\partial H_2}\\
    &= -\frac{\partial}{\partial H_2}\left[-k_bT\ln\left(e^{\beta J}2\cosh\left(\beta(H_1 + H_2)\right) + e^{-\beta J}2\cosh\left(\beta(-H_1 + H_2)\right)\right)\right]\\
    &= - k_bT\left[2\beta e^{-\beta J }\sinh\left(\beta[H_1 - H_2]\right) + \beta\right]\cdot\tanh\left[2e^{-\beta J}\cosh\left(\beta[H_2 - H_1]\right) + \beta(H_1 + H_2)\right].
\end{align*}One has that $H_1 = H_2 = H$ and thus:
\begin{align*}
    M &= \frac{1}{2}\left(\average{S_1} + \average{S_2}\right)\\
    &= -\frac{k_bT}{2}\left(\left[\beta\right]\tanh\left[2e^{-\beta J} + 2H\beta\right] + \left[\beta\right]\tanh\left[2e^{-\beta J} + 2H\beta\right]\right)\\
    &= -\tanh\left[2e^{-\beta J} + 2H\beta\right]
\end{align*}
\end{comment}

\subsection*{6)}
What is $M$ in the limit $T\to 0$?

\newparagraph
\textbf{Answer:}We can rewrite the limit as $\beta\to\infty$, which gives the following:
\begin{align*}
    \lim_{\beta\to\infty}M&= \lim_{\beta\to\infty}\left(\frac{1}{2}\frac{2 \sinh(2\beta H)}{e^{-J\beta} + 2\cosh(2\beta H)}\right)\\
    &= \lim_{\beta\to\infty}\left(\frac{\sinh(2\beta H)}{\cosh(2\beta H)}\right) = \lim_{\beta\to\infty}\left(\tanh(2\beta H)\right) = \mp 1,
\end{align*}The magnetization per unit spin $M$ is then given by $\mp 1$ in the limit $T\to 0$, depending on the sign of $H$.

\subsection*{7)}
Calculate $\average{S_1 \cdot S_2} - \average{S_1}\average{S_2}$ the connected correlation function.

\newparagraph
\textbf{Answer:} We begin by computing the average of $S_1$ and $S_2$:
\begin{align*}
    \average{S_1} &= \sum_i S_1^{(i)} \cdot p_i = \frac{1}{Z}\left(e^{\beta(2H + J)} + e^{-\beta J} - e^{-\beta J} - e^{\beta(-2H + J)}\right)\\
    &= \frac{e^{\beta(2H + J)} - e^{\beta(-2H + J)}}{e^{\beta J}\left[2e^{-2\beta J} + 2\cosh(2\beta H)\right]} = \frac{2\sinh(2H)}{\left[2e^{-2\beta J} + 2\cosh(2\beta H)\right]}\\
    \average{S_2} &= \sum_i S_2^{(i)} \cdot p_i = \frac{1}{Z}\left(e^{\beta(2H + J)} + e^{-\beta J} - e^{-\beta J} - e^{\beta(-2H + J)}\right)\\
    &= \frac{e^{\beta(2H + J)} - e^{\beta(-2H + J)}}{e^{\beta J}\left[2e^{-2\beta J} + 2\cosh(2\beta H)\right]} = \frac{2\sinh(2H)}{\left[2e^{-2\beta J} + 2\cosh(2\beta H)\right]}\\
\end{align*}Thus $\average{S_1} = \average{S_2}$. We now compute $\average{S_1\cdot S_2}$:
\begin{align*}
    \average{S_1\cdot S_2} &=\frac{1}{Z}\sum_{S_1}\sum_{S_2}\exp\left[\beta\left(H(S_1 + S_2) + K(S_1\cdot S_2)\right)\right]\\
    &= \frac{1}{Z}\left[\sum_{S_1}\exp\left(H(S_1 + 1) + J(S_1\cdot 1)\right) + \exp\left(H(S_1 - 1) - J(S_1 \cdot1)\right)\right]\\
    &= \frac{1}{Z}\left[\sum_{S_1}\exp\left(\beta(H(2) + J(1))\right) + \exp\left(\beta(H(-2) + J(1))\right) - 2e^{-\beta J}\right]\\
    &=\frac{e^{\beta J}}{Z}\left[e^{2\beta H} - 2e^{-2\beta J} + e^{-2\beta H}\right]\\
    &= \frac{2\cosh(2\beta H)-2e^{-2\beta J}}{2e^{-2\beta J} + 2\cosh(2\beta H)}.
\end{align*}Thus, the connected correlation function is given by:
\begin{align*}
    G(S_1,S_2) &= \average{S_1\cdot S_2} - \average{S_1}\average{S_2}\\
    &= \frac{2\cosh(2\beta H)-2e^{-2\beta J}}{2e^{-2\beta J} + 2\cosh(2\beta H)} - \left(\frac{\sinh(2H\beta)}{\left[2e^{-2\beta J} + 2\cosh(2\beta H)\right]}\right)^2\\
    &= \frac{\left(2\cosh(2\beta H)-2e^{-2\beta J}\right)\cdot(2e^{-2\beta J} + 2\cosh(2\beta H))}{(2e^{-2\beta J} + 2\cosh(2\beta H))^2} - \left(\frac{\sinh(2H\beta)}{\left[2e^{-2\beta J} + 2\cosh(2\beta H)\right]}\right)^2\\
    &= \frac{4\cosh^2\left(2H\beta\right) - 4 e^{-4\beta J}-\sinh^2\left(2H\beta\right)}{(2e^{-2\beta J} + 2\cosh(2\beta H))^2}\\
    &= \frac{4 - 4 e^{-4\beta J}}{(2e^{-2\beta J} + 2\cosh(2\beta H))^2}\\
    &= \frac{1-e^{-4\beta J}}{\left(e^{-2\beta J} + 2\cosh(2\beta H)\right)^2}.
\end{align*}

\subsection*{8)}
Show that, if we change the Hamiltonian of the system so that:
\begin{align*}
    -\mathcal{H} = H_1S_1 + H_2S_2 + JS_1S_2,
\end{align*}the connected correlation function can be written as the second derivative of the free energy.

\newparagraph
\textbf{Answer:} We first define the partition function:
\begin{align*}
    Z &= \sum_i e^{-\beta \mathcal{H}_i}.
\end{align*}The free energy is then given by:
\begin{align*}
    \mathcal{F} = -k_b T\ln(Z).
\end{align*}The correlation function, $G(S_1, S_2)$ can be written in the following manner:
\begin{align}
    G(S_1, S_2) &= \frac{1}{Z}\sum_{S_1}\sum_{S_2}S_1S_2e^{\beta \mathcal{H}} - \average{S_1}\average{S_2}.\label{eq: corr}
\end{align}We compute the two partial derivatives of $\mathcal{F}$ with respect to $H_1$ and $H_2$:
\begin{align}
    \frac{\partial^2\mathcal{F}}{\partial H_1\partial H_2} &= \frac{k_bT}{Z}\frac{\partial^2 Z}{\partial H_1\partial H_2}\nonumber\\
    &= \frac{k_bT}{Z}\left(\beta^2\sum_{S_1}\sum_{S_2}S_1S_2e^{\beta(H_1S_1 + H_2S_2 + JS_1S_2)}\right)\nonumber\\
    &= \frac{\beta}{Z} \left(\sum_{S_1}\sum_{S_2}S_1S_2e^{\beta\mathcal{H}_i}\right)\label{eq: part1}
\end{align}Comparing \eqref{eq: corr} and \eqref{eq: part1} we see that:
\begin{align*}
    G(S_1, S_2) &= \frac{1}{\beta}\frac{\partial^2\mathcal{F}}{\partial H_1\partial H_2} - \average{S_1}\average{S_2}.
\end{align*}And thus one has proved that the connected correlation function can be written as the second derivative of the free energy.

\begin{comment}
If we instead have the above Hamiltonian, we compute the partition function:
\begin{align*}
    Z &= \sum_i e^{-\beta E_i} = e^{\beta(H_1 + H_2 + J)} + e^{\beta(-H_1 + H_2 -J)} + e^{\beta(H_1 - H_2 - J)} + e^{\beta(-H_1 - H_2 + J)}\\
    &= e^{\beta J}\left[e^{\beta(H_1 + H_2)} + e^{-\beta(H_1 + H_2)}\right] + e^{-\beta J}\left[e^{\beta(-H_1 + H_2)} + e^{-\beta(-H_1 + H_2)}\right]\\
    &= e^{\beta J}2\cosh\left(\beta(H_1 + H_2)\right) + e^{-\beta J}2\cosh\left(\beta(-H_1 + H_2)\right).
\end{align*}The connected correlation function can still be written as:
\begin{align*}
    \average{S_1\cdot S_2} &= \frac{1}{Z}\sum_{S_1}\sum_{S_2}\exp\left[\beta\left(H_1S_1 + H_2S_2 + K(S_1\cdot S_2)\right)\right]\\
    &= \frac{1}{Z}\left[\sum_{S_1}\exp\left(\beta(H_1S_1 + H_2 + JS_1)\right) + \exp\left(\beta(H_1S_1 - H_2 + JS_1)\right)\right]\\
    &= \frac{1}{Z}\Bigg[\exp\left(\beta(H_1 + H_2 + J)\right) + \exp\left(\beta(-H_1 + H_2 - J)\right)\\
    & + \exp\left(\beta(H_1 - H_2 - J)\right) + \exp\left(\beta(-H_1 - H_2 + J)\right)\Bigg]\\
    &= \frac{1}{Z}\left[e^{\beta J}\left(e^{\beta(H_1 + H_2)} + e^{-\beta(H_1+H_2)}\right) + e^{-\beta J}\left(e^{\beta(-H_1 + H_2)}+e^{-\beta(-H_1 + H_2)}\right)\right]\\
    &=\frac{e^{\beta J}2\cosh\left(\beta(H_1 + H_2)\right) + e^{-\beta J}2\cosh\left(\beta(-H_1 + H_2)\right)}{e^{\beta J}2\cosh\left(\beta(H_1 + H_2)\right) + e^{-\beta J}2\cosh\left(\beta(-H_1 + H_2)\right)} = 1.
\end{align*}
\end{comment}

\end{document}
 
