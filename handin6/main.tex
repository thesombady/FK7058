\documentclass[a4paper]{article}

% --- Packages ---

\usepackage{a4wide}
\usepackage[utf8]{inputenc}
\usepackage{amsmath}
\usepackage{mathtools}
\usepackage{amssymb}
\usepackage[english]{babel}
\usepackage{mdframed}
\usepackage{systeme,}
\usepackage{lipsum}
\usepackage{relsize}
\usepackage{caption}
\usepackage{tikz}
\usepackage{tikz-3dplot}
\usetikzlibrary{shapes.geometric}
\usepackage{pgfplots}
\usepackage{pgfplotstable}
\pgfplotsset{compat=newest}%1.7}
\usepackage{harpoon}%
\usepackage{graphicx}
\usepackage{wrapfig}
\usepackage{subcaption}
\usepackage{authblk}
\usepackage{float}
\usepackage{listings}
\usepackage{xcolor}
\usepackage{chngcntr}
\usepackage{amsthm}
\usepackage{comment}
\usepackage{commath}
\usepackage{hyperref}%Might remove, adds link to each reference
\usepackage{url}
\usepackage{calligra}
\usepackage{pgf}

% --- Bibtex ---

%\usepackage[backend = biblar,]{bibtex}

%\addbibliografy(ref.bib)

% --- Commands --- 

\newcommand{\w}{\omega}
\newcommand{\trace}{\text{Tr}}
\newcommand{\grad}{\mathbf{\nabla}}
%\newcommand{\crr}{\mathfrak{r}}
\newcommand{\laplace}{\nabla^2}
\newcommand{\newparagraph}{\vspace{.5cm}\noindent}

% --- Math character commands ---

\newcommand{\curl}[1]{\mathbf{\nabla}\times \mathbf{#1}}
\newcommand{\dive}[1]{\mathbf{\nabla}\cdot \mathbf{#1}}
\newcommand{\res}[2]{\text{Res}(#1,#2)}
\newcommand{\fpartial}[2]{\frac{\partial #1}{\partial #2}}
\newcommand{\rot}[3]{\begin{vmatrix}\hat{x}&\hat{y}&\hat{z}\\\partial_x&\partial_y&\partial_z\\#1&#2&#3 \end{vmatrix}}
\newcommand{\average}[1]{\langle #1 \rangle}
\newcommand{\ket}[1]{|#1\rangle}
\newcommand{\bra}[1]{\langle #1|}


%  --- Special character commands ---

\DeclareMathAlphabet{\mathcalligra}{T1}{calligra}{m}{n}
\DeclareFontShape{T1}{calligra}{m}{n}{<->s*[2.2]callig15}{}
\newcommand{\crr}{\mathcalligra{r}\,}
\newcommand{\boldscriptr}{\pmb{\mathcalligra{r}}\,}


\title{Handin 6}
\author{Author : Andreas Evensen}
\date{Date: \today}

% --- Code ---

\definecolor{codegreen}{rgb}{0,0.6,0}
\definecolor{codegray}{rgb}{0.5,0.5,0.5}
\definecolor{codepurple}{rgb}{0.58,0,0.82}
\definecolor{backcolour}{rgb}{0.95,0.95,0.92}

\lstdefinestyle{mystyle}{
    backgroundcolor=\color{backcolour},   
    commentstyle=\color{codegreen},
    keywordstyle=\color{magenta},
    numberstyle=\tiny\color{codegray},
    stringstyle=\color{codepurple},
    basicstyle=\ttfamily\footnotesize,
    breakatwhitespace=false,         
    breaklines=true,                 
    captionpos=b,                    
    keepspaces=true,                 
    numbers=left,                    
    numbersep=5pt,                  
    showspaces=false,                
    showstringspaces=false,
    showtabs=false,                  
    tabsize=2
}

\lstset{style=mystyle}

\begin{document}

\maketitle

\noindent
Consider the free energy density:
\begin{align*}
    \mathcal{L} &= \frac{1}{2}a\eta^2 + \frac{1}{4}b\eta^4 + \frac{1}{6}c\eta^6 - h\eta,
\end{align*}where $ c> 0 $, $a$ and $b$ are both linearly proportional to the pressure $p$ and temperature $T$ near the critical point, $(T_c, p_c)$.
\begin{align*}
    a &= a_1t + a_2p,\quad b = b_1t + b_2p.
\end{align*}Here, $t$ and $p$ are the reduced temperature and pressure, respectively.
As $T$ and $p$ are varied both $a$ and $b$ can be made to vanish or change sign.
Such a system exhibits a tricritical point. An example of this is $He^3$ and $He^4$.
In this question we will calculate the phase diagram in the $a-b$ plane for $h = 0$, using Landau theory.

\section*{a)}
Consider the case $a<0$. Find the extrema for the Landau free energy, and study their stability. Hence, show:
\begin{align*}
    \average{\eta}^2=\eta_s^2 = \frac{-b + \sqrt{b^2 - 4ac}}{2c}.
\end{align*}

\newparagraph
\textbf{Answer: } Since $h = 0$, we're left with the following Landau free energy density (LFD):
\begin{align}
    \mathcal{L} &= \frac{1}{2}a\eta^2 + \frac{1}{4}b\eta^4 + \frac{1}{6}c\eta^6. \label{eq: LFD}
\end{align}


\section*{b)}
Now consider $a>0$, $b>0$. How does $\eta_s$ behave in this region?

\newparagraph
\textbf{Answer: }
\section*{c)}
Lastly, consider $a>0$, $b<0$. What happens here?

\newparagraph
\textbf{Answer: }


\section*{d)}
Now sketch the phase diagram in the $a-b$ plane, indicating the order of any phase transition that you have found, and the position of the phase boundaries.
Sketch the form of the Landau free energy in each region.
The point $(a,b) = (0,0)$ is the tricritical point.
Can you suggest why? \textit{Hint:} Consider $h \neq 0$.


\newparagraph
\textbf{Answer: }


\section*{e)}
Calculate the thermodynamic critical exponent, by approaching the tricritical point along the line $b = 0$.
What do you expect $\nu$ to be?

\newparagraph
\textbf{Answer: }


\section*{f)}
Show that for small but positive $b$, there is a crossover from the tricritical behavior which you have found to ordinary critical behavior when $b^2\sim - ac$.

\newparagraph
\textbf{Answer: }



\end{document}
 
