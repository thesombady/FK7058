\documentclass[a4paper]{article}

% --- Packages ---

\usepackage{a4wide}
\usepackage[utf8]{inputenc}
\usepackage{amsmath}
\usepackage{mathtools}
\usepackage{amssymb}
\usepackage[english]{babel}
\usepackage{mdframed}
\usepackage{systeme,}
\usepackage{lipsum}
\usepackage{relsize}
\usepackage{caption}
\usepackage{tikz}
\usepackage{tikz-3dplot}
\usetikzlibrary{shapes.geometric}
\usepackage{pgfplots}
\usepackage{pgfplotstable}
\pgfplotsset{compat=newest}%1.7}
\usepackage{harpoon}%
\usepackage{graphicx}
\usepackage{wrapfig}
\usepackage{subcaption}
\usepackage{authblk}
\usepackage{float}
\usepackage{listings}
\usepackage{xcolor}
\usepackage{chngcntr}
\usepackage{amsthm}
\usepackage{comment}
\usepackage{commath}
\usepackage{hyperref}%Might remove, adds link to each reference
\usepackage{url}
\usepackage{calligra}
\usepackage{pgf}

% --- Bibtex ---

%\usepackage[backend = biblar,]{bibtex}

%\addbibliografy(ref.bib)

% --- Commands --- 

\newcommand{\w}{\omega}
\newcommand{\trace}{\text{Tr}}
\newcommand{\grad}{\mathbf{\nabla}}
%\newcommand{\crr}{\mathfrak{r}}
\newcommand{\laplace}{\nabla^2}
\newcommand{\newparagraph}{\vspace{.5cm}\noindent}

% --- Math character commands ---

\newcommand{\curl}[1]{\mathbf{\nabla}\times \mathbf{#1}}
\newcommand{\dive}[1]{\mathbf{\nabla}\cdot \mathbf{#1}}
\newcommand{\res}[2]{\text{Res}(#1,#2)}
\newcommand{\fpartial}[2]{\frac{\partial #1}{\partial #2}}
\newcommand{\rot}[3]{\begin{vmatrix}\hat{x}&\hat{y}&\hat{z}\\\partial_x&\partial_y&\partial_z\\#1&#2&#3 \end{vmatrix}}
\newcommand{\average}[1]{\langle #1 \rangle}
\newcommand{\ket}[1]{|#1\rangle}
\newcommand{\bra}[1]{\langle #1|}


%  --- Special character commands ---

\DeclareMathAlphabet{\mathcalligra}{T1}{calligra}{m}{n}
\DeclareFontShape{T1}{calligra}{m}{n}{<->s*[2.2]callig15}{}
\newcommand{\crr}{\mathcalligra{r}\,}
\newcommand{\boldscriptr}{\pmb{\mathcalligra{r}}\,}


\title{Handin 3}
\author{Author : Andreas Evensen}
\date{Date: \today}

% --- Code ---

\definecolor{codegreen}{rgb}{0,0.6,0}
\definecolor{codegray}{rgb}{0.5,0.5,0.5}
\definecolor{codepurple}{rgb}{0.58,0,0.82}
\definecolor{backcolour}{rgb}{0.95,0.95,0.92}

\lstdefinestyle{mystyle}{
    backgroundcolor=\color{backcolour},   
    commentstyle=\color{codegreen},
    keywordstyle=\color{magenta},
    numberstyle=\tiny\color{codegray},
    stringstyle=\color{codepurple},
    basicstyle=\ttfamily\footnotesize,
    breakatwhitespace=false,         
    breaklines=true,                 
    captionpos=b,                    
    keepspaces=true,                 
    numbers=left,                    
    numbersep=5pt,                  
    showspaces=false,                
    showstringspaces=false,
    showtabs=false,                  
    tabsize=2
}

\lstset{style=mystyle}

\begin{document}

\maketitle


\section*{Problem 1}
Consider the one-dimensional Ising-model.

\subsection*{a)}
In the thermodynamics limit and $H = 0$, write an expression of the free energy $\mathcal{F}$. Determine the low and high temperature limits of $\mathcal{F}$ and explain what they imply.

\newparagraph
\textbf{Answer:} The partition function for the Ising model is given by:
\begin{align*}
    Z &= \sum_{\{S\}}\exp\left[h\sum_i S_i + K\sum_{\average{ij}}S_iS_j\right],
\end{align*}and since $H = 0$, the first term is zero.
\begin{align*}
    Z &= \sum_{\{S\}}\exp\left[K\sum_{\average{ij}}S_iS_j\right].
\end{align*}One rewrite the second sum in the following manner,
\begin{align*}
    Z &= \sum_{\{S\}}\exp\left[K\sum_{i = 1}^{N - 1}S_iS_{i+1}\right]\\
    &= 2\left(2\cosh(K)\right)^{N-1}.
\end{align*}The free energy is then given by:
\begin{align*}
    \mathcal{F} &= -k_bT\ln(Z)\\
    &= -k_bT\ln\left(2\left(2\cosh(K)\right)^{N-1}\right)\\
    &= -k_bT\left[\ln(2) + (N-1)\ln\left(2\cosh(K)\right)\right]\\
    &= -k_bT\left[\ln(2) + (N-1)\ln\left(e^{K} + e^{-K}\right)\right]\\
    &= -k_bT\left[\ln(2) + (N-1)\ln\left[e^K\left(1 + e^{-2K}\right)\right]\right]\\
    &= -k_bT\left[\ln(2) + (N-1)\left(K + \ln\left(1 + e^{-2K}\right)\right)\right].
\end{align*}The low and high temperature limits are given by:
\begin{align*}
    &\lim_{T\to0}\lim_{N\to\infty}\left[-k_bT\left[\ln(2) + (N-1)\left(K + \ln\left(1 + e^{-2K}\right)\right)\right]\right]\\
    &= \lim_{T\to0}\lim_{N\to\infty}\left[-k_bT\left[N\left(\frac{J}{k_bT} + \ln\left(1 + e^{-\frac{2J}{k_bT}}\right)\right)\right]\right]\\
    &= \lim_{T\to0}\lim_{N\to\infty}\left[-k_bT\left[N\left(\frac{J}{k_bT} + \ln\left(1 + e^{-\frac{2J}{k_bT}}\right)\right)\right]\right]\\
    &= \lim_{N\to\infty}\left[-JN\right] = -\infty,\\
    &\lim_{T\to\infty}\lim_{N\to\infty}\left(-k_bT\left[\ln(2) + (N-1)\left(K + \ln\left(1 + e^{-2K}\right)\right)\right]\right) = -\infty,\\
\end{align*}In the thermodynamic limit, the free energy diverges.
\subsection*{b)}
Obtain expressions for the average energy $\average{E}$ and the heat-capacity $C$ using the expression of $\mathcal{F}$.

\newparagraph
\textbf{Answer:} The average energy is defined by:
\begin{align*}
    \average{E} &= -\frac{1}{Z}\fpartial{Z}{\beta}\\
    &=-\frac{1}{\left(2\cosh(J\beta)\right)^{N-1}}\frac{\partial}{\partial \beta}\left[\left(2\cosh(J\beta)\right)^{N-1}\right]\\
    &= -\frac{2}{\left(2\cosh(J\beta)\right)^{N-1}}\left(N-1\right)J\sinh(J\beta)\left(2\cosh(J\beta)\right)^{N-2}\\
    &= -J\left(N-1\right)\tanh(J\beta).
\end{align*}The heat capacity is defined by:
\begin{align*}
    C &= \fpartial{\average{E}}{T}\\
    &= \fpartial{}{\beta}\left(-J\left(N-1\right)\tanh\left(\frac{J}{k_bT}\right)\right)\\
    &= \frac{J^2}{k_bT^2}\frac{(N-1)}{\cosh^2\left(\frac{J}{k_bT}\right)}.
\end{align*}

\subsection*{c)}
Consider now the Ising model in a field and write an expression for $\mathcal{F}$ (again in the thermodynamic limit).

\newparagraph
\textbf{Answer:} Now $H$ is non-zero, and thus the partition function is given by:
\begin{align*}
    Z &= \sum_{\{S\}}\exp\left[h\sum_i S_i + K\sum_{\average{ij}}S_iS_j\right].
\end{align*}Furthermore, with introducing the transfer matrix, one obtains that the free energy can be written as:
\begin{align*}
    \mathcal{F} &= -k_bTN\ln\left[e^{K}\left(\cosh(h) + \sqrt{\sinh^2(h) + e^{-4K}}\right)\right]\\
    &= -JN - k_bTN\ln\left[\cosh(h) + \sqrt{\sinh^2(h) + e^{-4K}}\right].
\end{align*}

\subsection*{d)}
Compute the magnetization per spin $M$. What is $M$ in the limit $T\to0$?

\newparagraph
\textbf{Answer:} One computes the following expression for the magnetization per spin:
\begin{align*}
    M &= \frac{1}{N}\fpartial{\mathcal{F}}{H}\\
    &= \frac{1}{k_bT}\fpartial{}{h}\left(- J - k_bT\ln\left[\cosh(h) + \sqrt{\sinh^2(h) + e^{-4K}}\right]\right)\\
    &= \frac{\sinh^2(h)}{\cosh^2(h) + e^{-4K}}.
\end{align*}
\section*{Problem 2}
Consider the $d = 1$ Ising-model with periodic boundary conditions.

\subsection*{a)}
Construct the matrix $\mathbf{S}$ which diagonalises the transfer matrix $\mathbf{T}$.
You will find it useful to write the matrix elements in terms of the variable $\theta$ given by:
\begin{align*}
    \coth(2\theta) = e^{2K}\sinh(h).
\end{align*}

\newparagraph
\textbf{Answer:} Firstly, we define the transfer matrix $\mathbf{T}$ as:
\begin{align*}
    \mathbf{T} = \begin{pmatrix}
        e^{K + h} & e^{-K}\\
        e^{-K}\ & e^{-h + K}
    \end{pmatrix}.
\end{align*}A matrix $\mathbf{S}$ that diagonalises $\mathbf{T}$ is given by:
\begin{align}
    \mathbf{T}' &= \mathbf{S}^{-1}\mathbf{T}\mathbf{S}\\
    \begin{pmatrix}
        \lambda_1 & 0\\
        0 & \lambda_2
    \end{pmatrix} &= \mathbf{S}^{-1}\begin{pmatrix}
        e^{K + h} & e^{-K}\\
        e^{-K}\ & e^{-h + K}
    \end{pmatrix}\mathbf{S}.\label{eq: diagonalise}
\end{align}One needs to find the eigenvalues of $\mathbf{T}$, which is achieved by solving the characteristic equation:
\begin{align*}
    \det\left(T - \lambda I\right) &= \begin{vmatrix}
        e^{K + h} - \lambda & e^{-K}\\
        e^{-K}\ & e^{-h + K} - \lambda
    \end{vmatrix}\\
    &=\left(e^{K + h} - \lambda\right)\cdot\left(e^{K - h} - \lambda\right) - e^{-2K}\\
    &= \lambda^2 - \left(e^{K + h} + e^{K - h}\right)\lambda + e^{2K} - e^{-2K}\\
\end{align*}The roots of this equation are given by:
\begin{align*}
    \lambda_{1,2} &= \frac{e^{K + h} + e^{K - h}}{2} \pm \sqrt{\left(\frac{e^{K + h} + e^{K - h}}{2}\right)^2 - e^{2K} + e^{-2K}}\\
    &= e^{K}\left[\cosh(h) \pm \sqrt{\sinh^2(h) + e^{-4K}}\right].
\end{align*}Thus, $\lambda_1$ and $\lambda_2$ are the eigenvalues of $\mathbf{T}$, i.e. $\lambda_{1,2} = e^{K}\left[\cosh(h) \pm \sqrt{\sinh^2(h) + e^{-4K}}\right]$.
Furthermore, we seek the eigenvectors corresponding to the eigenvalues of $\mathbf{T}$. The eigenvectors are given by:
\begin{align*}
    (\lambda_1 \mathbf{I} - \mathbf{T})\mathbf{v}_1 &= \mathbf{0},\\
    (\lambda_2 \mathbf{I} - \mathbf{T})\mathbf{v}_2 &= \mathbf{0}.
\end{align*}Looking at the first equation, one obtains:
\begin{align*}
    \begin{pmatrix}
        \lambda_1-e^{K + h} & -e^{-K}\\
        -e^{-K}\ & \lambda_1-e^{-h + K}
    \end{pmatrix}\begin{pmatrix}
        v_{11}\\
        v_{12}
    \end{pmatrix}
    &= \begin{pmatrix}
        0\\
        0
    \end{pmatrix}\\
    v_{11}\left(\lambda_1 - e^{K + h}\right) - v_{12} e^{-K} &= 0\\
    v_{12}\left(\lambda_1 - e^{-h + K}\right) - v_{11} e^{-K} &= 0
\end{align*}This implies that one can write the eigenvector of $\lambda_1$ as:
\begin{align*}
    \mathbf{v}_1 &= \begin{pmatrix}
        v_{11}\\
        v_{12}
    \end{pmatrix} = \begin{pmatrix}
        e^{-K}\\
        -e^K\sinh(h)+\sqrt{e^{2K}\sinh^2(h) + e^{-2k}}
    \end{pmatrix},
\end{align*}and similarly, the eigenvector of $\lambda_2$ is given by:
\begin{align*}
    \mathbf{v}_2 &= \begin{pmatrix}
        v_{21}\\
        v_{22}
    \end{pmatrix} = \begin{pmatrix}
        -e^K\sinh(h)+\sqrt{e^{2K}\sinh^2(h) + e^{-2k}}\\
        -e^{-K}\\
    \end{pmatrix}.
\end{align*}
Using $\cot(2\theta) = e^{2K}\sinh(h)$, on the term $-e^{k}\sinh(h) + \sqrt{e^{2K}\sinh^2(h) + e^{-2K}}$, one gets:
\begin{align*}
    -e^{k}\sinh(h) + \sqrt{e^{2K}\sinh^2(h) + e^{-2K}} &= e^{-K}\left(-e^{2K}\sinh(h) + \sqrt{\cot^2(2\theta) + 1}\right)\\
    &=e^{-K}\tan(\theta).
\end{align*}Thus, the matrix $\mathbf{S}$ can be written as, after normalizing the eigenvectors, i.e $\mathbf{v_{i}}/\norm{\mathbf{v_{i}}}$:
\begin{align*}
    \mathbf{S} &=\begin{pmatrix}
        \cos(\theta) & -\sin(\theta)\\
        \sin(\theta) &\cos(\theta)
    \end{pmatrix}.
\end{align*}

\begin{comment}
    \implies \lambda_1 - e^{K + h} v_{11} - e^{-K} v_{12} &= 0,\\
    \lambda_1 - e^{-h + K} v_{12} - e^{-K} v_{11} &= 0\\
    \implies \lambda_1 -  \frac{\lambda_1 e^{-h + K} v_{12}}{e^{-K}} - e^{-K} v_{12} &=0\\
    \implies v_{12} = \frac{\lambda_1}{\left(\lambda_1 e^{-h + 2K} + e^{-K}\right)}&\\
    \implies v_{11} = \frac{\lambda_1}{e^{-K}}\left(1 - \frac{1}{\left(\lambda_1 e^{-h + 2K} + e^{-K}\right)}\right)&
\end{align*}Now looking at the term $\lambda_1e^{-h + 2K}$, one can rewrite it as:
\begin{align*}
    \lambda_1 e^{-h + 2K} &= e^{3K -h}\left[\cosh(h) + \sqrt{\sinh^2\left(h\right) + e^{-4K}}\right]\\
    &=e^{K - h}\left[\cosh(h)e^{2K} + \sqrt{\coth^2(2\theta) + 1}\right]
\end{align*}One does similar with the second equation, and obtains:
\begin{align*}
    v_{21} &= \frac{\lambda_2}{e^{-K}}\left(1 - \frac{1}{\left(\lambda_2 e^{-h + 2K} + e^{-K}\right)}\right),\\
    v_{22} &= \frac{\lambda_2}{\left(\lambda_2 e^{-h + 2K} + e^{-K}\right)}.
\end{align*}Thus, the matrix $\mathbf{S}$ is given by:
\begin{align*}
    \mathbf{S} &= \begin{pmatrix}
        v_{11} & v_{21}\\
        v_{12} & v_{22}
    \end{pmatrix}.
\end{align*}
\end{comment}
\begin{comment}
Multiplying, from the right, with $\mathbf{S}$, one obtains:
\begin{align*}
    \mathbf{S}\begin{pmatrix}
        \lambda_1 & 0\\
        0 & \lambda_2
    \end{pmatrix}
    &= \begin{pmatrix}
        e^{K + h} & e^{-K}\\
        e^{-K}\ & e^{-h + K}
    \end{pmatrix}\mathbf{S}.
\end{align*}Labeling the elements of $\mathbf{S}$ as $s_{ij}$, and multiplying the matrices, one obtains:
\begin{align*}
    \begin{pmatrix}
        \lambda_1 s_{11} & \lambda_2 s_{12}\\
        \lambda_1 s_{21} & \lambda_2 s_{22}
    \end{pmatrix} &= \begin{pmatrix}
        e^{K + h}s_{11} + e^{-K}s_{21} & e^{K + h}s_{12} + e^{-K}s_{22}\\
        e^{-K}s_{11} + e^{-h + K}s_{21} & e^{-K}s_{12} + e^{-h + K}s_{22}
    \end{pmatrix}.
\end{align*}From this, each element of the two matrices are equal, and thus one obtains the following equations, and substitute $e^{2K}\sinh(h) = \coth(2\theta)$:
\begin{align*}
    \left[e^{K}\cosh(h) + \sqrt{e^{2K}\left(\sinh^2(h) + e^{-4K}\right)}\right] s_{11} &= e^{K + h}s_{11} + e^{-K}s_{21},\\
    \left[e^{K}\cosh(h) - \sqrt{e^{2K}\left(\sinh^2(h) + e^{-4K}\right)}\right] s_{12} &= e^{K + h}s_{12} + e^{-K}s_{22},\\
    \left[e^{K}\cosh(h) + \sqrt{e^{2K}\left(\sinh^2(h) + e^{-4K}\right)}\right] s_{21} &= e^{-K}s_{11} + e^{-h + K}s_{21},\\
    \left[e^{K}\cosh(h) - \sqrt{e^{2K}\left(\sinh^2(h) + e^{-4K}\right)}\right] s_{22} &= e^{-K}s_{12} + e^{-h + K}s_{22}.
\end{align*}
From this, one has two sets of equations:
\begin{align}
    \lambda_1 s_{11} + \lambda_2 s_{12} &= e^{K + h}s_{11} + e^{-K}s_{21} + e^{K + h}s_{12} + e^{-K}s_{22},\label{eq: expansion1},\\
    \lambda_1 s_{21} + \lambda_2 s_{22} &= e^{-K}s_{11} + e^{K - h}s_{21} + e^{-K}s_{12} + e^{K-h}s_{22}\label{eq: expansion2}.
\end{align}Expanding equation \eqref{eq: expansion1}, with the corresponding eigenvalues, and divided by $e^{K}$, one obtains:
\begin{align*}
    s_{11}\left[\cosh(h) + \sqrt{\sinh^2(h) + e^{-4K}}\right] + s_{12}\left[\cosh(h) - \sqrt{\sinh^2(h) + e^{-4K}}\right] &= e^{h}s_{11} + e^{-2K}s_{21} + e^{h}s_{12} + e^{-2K}s_{22}.
\end{align*}
\end{comment}


\subsection*{b)}
Derive the relation:
\begin{align*}
    \average{S_i} &= \frac{\trace\left(\mathbf{S}^{-1}\sigma_z\mathbf{S}\mathbf{T}'^N\right)}{Z_N},
\end{align*}and use your answer from part a) to show that $\average{S_i} = \cos(2\theta)$ as $N\to\infty$.
Similarly compute $\average{S_iS_j}$ and hence show that in the thermodynamics limit:
\begin{align*}
    G(i, i + j) &=\average{S_iS_j} - \average{S_i}\average{S_j}\\
    &=\sin^2\left(2\theta\right)\left(\frac{\lambda_2}{\lambda_1}\right)^j.
\end{align*}

\newparagraph
\textbf{Answer:} We begin by stating that the average spin of a particle $i$ is given by:
\begin{align*}
    \average{S_i} &=\frac{1}{Z_n}\sum_{S_1}...\sum_{S_{N}}T_{S_1, S_2}...T_{S_{i-1}, S_{i}}S_i... T_{S_{N-1}, S_N}\\
    &= \frac{1}{Z_n}\sum_{S_1}...\sum_{S_N}...\mathbf{T}\underbrace{\begin{pmatrix}1&0\\0&-1\end{pmatrix}}_{=\sigma_z}\mathbf{T}...\mathbf{T}\sigma_z\mathbf{T},
\end{align*}This then yields that the average spin of particle $i$ can be written as:
\begin{align*}
    \average{S_i} &=\frac{\trace\left(\sigma_z\mathbf{T}^N\right)}{Z_n}.
\end{align*}From this, one then inverts eq \eqref{eq: diagonalise}, such that $\mathbf{T} = \mathbf{S}\mathbf{T}'\mathbf{S}^{-1}$, and thus:
\begin{align*}
    \average{S_i} &= \frac{\trace\left(\sigma_z \left[\mathbf{S}\mathbf{T}'\mathbf{S}^{-1}\right]^N\right)}{Z_n}.
\end{align*}From this, one expands the $(\mathbf{S}T'\mathbf{S}^{-1})^N$ term and utilizes that $\mathbf{S}^{-1}\cdot\mathbf{S} = \mathbf{I}$, such that one obtain:
\begin{align*}
    \average{S_i} &= \frac{\trace\left(\sigma_z\mathbf{S}\mathbf{T}'\mathbf{S}^{-1}\mathbf{S}T'...\mathbf{S}T'\mathbf{S}^{-1}\right)}{Z_N}\\
    \average{S_i} &= \frac{\trace\left(\mathbf{S}^{-1}\sigma_z\mathbf{S}\mathbf{T}'^N\right)}{Z_N}.
\end{align*}In order to evaluate $\average{S_i}$ in the thermodynamic limit, one has to evaluate the expression $\mathbf{S}^{-1}\sigma_z\mathbf{S}$
\begin{align*}
    \mathbf{S}^{-1}\sigma_z\mathbf{S} &= \begin{pmatrix}
        \cos(\theta) & \sin(\theta)\\
        -\sin(\theta) &\cos(\theta)
    \end{pmatrix}\cdot\begin{pmatrix}
        1 & 0 \\
        0 & -1
    \end{pmatrix}\cdot\begin{pmatrix}
        \cos(\theta) & -\sin(\theta)\\
        \sin(\theta) &\cos(\theta)
    \end{pmatrix}\\ &= \begin{pmatrix}
        \cos(2\theta) & -\sin(2\theta)\\
        -\sin(2\theta) &-\cos(2\theta)
    \end{pmatrix}.
\end{align*}Thus, $\average{S_i}$ can be expressed as:
\begin{align*}
    \average{S_i} &= \frac{\trace\left(\mathbf{S}^{-1}\sigma_z\mathbf{S}\mathbf{T}'^N\right)}{Z_N}\\
    &= \frac{\trace\left(\underbrace{\begin{pmatrix}
        \cos(2\theta) & -\sin(2\theta)\\
        -\sin(2\theta) &-\cos(2\theta)
    \end{pmatrix}}_{=\mathbf{\delta}}\cdot\begin{pmatrix}
        \lambda_1^N & 0\\
        0 & \lambda_2^N
    \end{pmatrix}\right)}{\trace\left(T^N\right)}
\end{align*}In the thermodynamic limit, this then simplifies to:
\begin{align*}
    \average{S_i} &= \cos(2\theta).
\end{align*}The correlation term $\average{S_iS_j}$ is then given by:
\begin{align*}
    \average{S_iS_j} &= \frac{1}{\trace\left(T^N\right)}\left(\mathbf{T}'^i\cdot \mathbf{\delta}\mathbf{T}'^j\mathbf{\delta}\mathbf{T}'^{N - i - j}\right).
\end{align*}This, in the thermodynamic limit, can be written as:
\begin{align*}
    \average{S_iS_j} &= \cos^2(2\theta) + \sin^2(2\theta)\left(\frac{\lambda_2}{\lambda_1}\right)^j.
\end{align*}Finally, the correlation function $G(i, i + j)$ is given by:
\begin{align*}
    G(i, i + j) &=\average{S_iS_j} - \average{S_i}\average{S_j}\\
    &=\cos^2(2\theta) + \sin^2\left(2\theta\right)\left(\frac{\lambda_2}{\lambda_1}\right)^j - \cos^2\left(2\theta\right)\\
    &= \sin^2\left(2\theta\right)\left(\frac{\lambda_2}{\lambda_1}\right)^j.
\end{align*}

\subsection*{c)}
Calculate the isothermal suspectability $\chi_T$ from the formula given for $M(H)$ in the text.
Verify explicitly that $\chi_T = \beta\sum_j G(i, i+j)$ (in the thermodynamic limit the sum runs from $-\infty\to\infty)$.

\newparagraph
\textbf{Answer:} The isothermal susceptibility is defined by:
\begin{align*}
    \chi_T &= \fpartial{M}{H}.
\end{align*}The magnetization per spin is given by:
\begin{align*}
    M &= \sum_{i = 1}^N\average{S_i} = N\cos(2\theta).
\end{align*}Thus, the isothermal susceptibility is given by:
\begin{align*}
    \chi_T &= \fpartial{M}{H} = N\fpartial{\cos(2\theta)}{H}\\
    &= N\beta\fpartial{\cos(2\theta)}{h}\\
    &= -N\beta\sin(2\theta)\fpartial{2\theta}{h}.
\end{align*}We now need to find an explicit expression for $\theta$ which one does from $\cot(2\theta) = e^{2K}\sinh(h)$, such that:
\begin{align*}
    \theta &= \frac{1}{2}\cot^{-1}\left(e^{2K}\sinh(h)\right).
\end{align*}This then yields:
\begin{align*}
    \chi_T &=-N\beta\sin(2\theta)\fpartial{\theta}{h}\\
    &= -2N\beta\sin\left(2\theta\right)\fpartial{}{h}\left(\frac{1}{2}\cot^{-1}\left(e^{2K}\sinh(h)\right)\right)\\
    &= N\beta\sin^3\left(2\theta\right)e^{2K}\cosh(h).
\end{align*}
To prove that the magnetic susceptibility is given by $\chi_T = \sum_j G(i, i+j)$, one needs to evaluate the sum:
\begin{align*}
    \sum_j G(i, i+j) &= \sum_j \sin^2\left(2\theta\right)\left(\frac{\lambda_2}{\lambda_1}\right)^j\\
    &= N\sin^2\left(2\theta\right)\left(\frac{\lambda_1 + \lambda_2}{\lambda_1 - \lambda_2}\right)\\
    &= N\sin^2\left(2\theta\right)\left(\frac{e^{-K}\sqrt{\cot^2(2\theta) + 1}}{e^{K}\cosh(h)}\right)^{-1}\\
    &= N\sin^2\left(2\theta\right)\left(\sin(2\theta)e^{2K}\cosh(h)\right)\\
    &= N\sin^3\left(2\theta\right)e^{2K}\cosh(h).
\end{align*}Since the $\beta G(i, i+j) = \chi_T$ one has proved that the identify holds.


\subsection*{d)} 
We will now consider what happens at the boundaries. Consider the partition function:
\begin{align*}
    Z(h, K) &= \sum_{S_1}\cdot...\sum_{S_N} \exp\left[h\left(S_1 + ...+ S_N\right) + K\left(S_1S_2 + ... + S_{N-1}S_N\right)\right].
\end{align*}In this case, the partition function is not simply $Z = \trace\left(\mathbf{T}'\right)^N$.
Work out what the correct expression is, you will have to introduce a new matrix in addition to $\mathbf{T}$, and show that the free energy is given by:
\begin{align*}
    \mathcal{F} &= N f_b(h, K) + f_s(h, K) + F_{f_s}(N, h, K),
\end{align*}where $f_b$ is the bulk free energy, $f_s$ is the surface free energy and $F_{f_s}$ is an intrinsic finite size contribution that depends on the system size as $e^{-C(h, K)N}$, where $C$ is function.

\newparagraph
\textbf{Answer:}
\subsection*{e)}
Check that in the case $h = 0$ and $N\to\infty$, your result for the surface free energy agrees with that obtained from $\lim_{N\to\infty}\mathcal{F}_N^{free}- \mathcal{F}_N^{periodic}$.

\newparagraph
\textbf{Answer:}




\end{document}
 
