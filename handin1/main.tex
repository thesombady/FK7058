\documentclass[a4paper]{article}
\usepackage{a4wide}
\usepackage[utf8]{inputenc}
\usepackage{amsmath}
\usepackage{mathtools}
\usepackage{amssymb}
\usepackage[english]{babel}
\usepackage{mdframed}
\usepackage{systeme,}
\usepackage{lipsum}
\usepackage{relsize}
\usepackage{caption}
\usepackage{tikz}
\usepackage{tikz-3dplot}
\usetikzlibrary{shapes.geometric}
\usepackage{pgfplots}
\usepackage{pgfplotstable}
\pgfplotsset{compat=newest}%1.7}
\usepackage{harpoon}%
\usepackage{graphicx}
\usepackage{wrapfig}
\usepackage{subcaption}
\usepackage{authblk}
\usepackage{float}
\usepackage{listings}
\usepackage{xcolor}
\usepackage{chngcntr}
\usepackage{amsthm}
\usepackage{comment}
\usepackage{commath}
\usepackage{hyperref}%Might remove, adds link to each reference
\usepackage{url}
\usepackage{calligra}

% Commands

\newcommand{\w}{\omega}
\newcommand{\curl}[1]{\mathbf{\nabla}\times \mathbf{#1}}
\newcommand{\grad}{\mathbf{\nabla}}
\newcommand{\dive}[1]{\mathbf{\nabla}\cdot \mathbf{#1}}
%\newcommand{\crr}{\mathfrak{r}}
\newcommand{\res}[2]{\text{Res}(#1,#2)}
\newcommand{\laplace}{\nabla^2}
\newcommand{\trace}{\text{Tr}}
\newcommand{\fpartial}[2]{\frac{\partial #1}{\partial #2}}
\newcommand{\rot}[3]{\begin{vmatrix}\hat{x}&\hat{y}&\hat{z}\\\partial_x&\partial_y&\partial_z\\#1&#2&#3 \end{vmatrix}}

% Special character commands
\DeclareMathAlphabet{\mathcalligra}{T1}{calligra}{m}{n}
\DeclareFontShape{T1}{calligra}{m}{n}{<->s*[2.2]callig15}{}
\newcommand{\crr}{\mathcalligra{r}\,}
\newcommand{\boldscriptr}{\pmb{\mathcalligra{r}}\,}
\newcommand{\average}[1]{\langle #1 \rangle}
\newcommand{\newparagraph}{\vspace{.5cm}\noindent}

\newcommand{\f}{\mathcal{F}}



\title{Statistical physics: Handin 1}
\author{Author : Andreas Evensen}
\date{Date: \today}
\definecolor{codegreen}{rgb}{0,0.6,0}
\definecolor{codegray}{rgb}{0.5,0.5,0.5}
\definecolor{codepurple}{rgb}{0.58,0,0.82}
\definecolor{backcolour}{rgb}{0.95,0.95,0.92}

\lstdefinestyle{mystyle}{
    backgroundcolor=\color{backcolour},   
    commentstyle=\color{codegreen},
    keywordstyle=\color{magenta},
    numberstyle=\tiny\color{codegray},
    stringstyle=\color{codepurple},
    basicstyle=\ttfamily\footnotesize,
    breakatwhitespace=false,         
    breaklines=true,                 
    captionpos=b,                    
    keepspaces=true,                 
    numbers=left,                    
    numbersep=5pt,                  
    showspaces=false,                
    showstringspaces=false,
    showtabs=false,                  
    tabsize=2
}

\lstset{style=mystyle}

\begin{document}

\maketitle

\section*{Task 1}
Given a harmonic oscillator, with angular frequency $\w_0$; with the classical definition of $H$:
\begin{align*}
    H = \frac{p^2}{2m} + \frac{1}{2}m\w_0^2x^2.
\end{align*}

\subsection*{a)}
Calculate the Canonical partition function $Z$ for the harmonic oscillator.

\vspace*{0.5cm}\noindent
The partition function, in a continuous canonical ensemble, is given by:
\begin{align*}
    Z &= \frac{1}{h}\int_{-\infty}^\infty dq \int_{-\infty}^{\infty} dp\left[ e^{-\beta H(q,p)}\right]\\
    &= \frac{1}{h}\int_{-\infty}^{\infty} dx \int_{-\infty}^\infty dp\left[ \exp\left\{-\beta\left[\frac{p^2}{2m} + \frac{1}{2}m\w_0^2x^2\right]\right\}\right]\\
    &= \frac{1}{h}\int_{-\infty}^{\infty} dx \exp\left[-\frac{\beta m\w_0^2x^2}{2}\right]\int_{-\infty}^{\infty} dp\exp\left[-\beta\left[\frac{p^2}{2m}\right]\right]\\
    &= \frac{1}{h}\left(\frac{2\pi}{\beta\w_0}\right).
\end{align*}

\subsection*{b)}
Calculate the average energy, $\langle E \rangle$ and the specific heat capacity.

\vspace*{0.5cm}\noindent
The average energy is given by:
\begin{align*}
    \average{E} &= -\frac{1}{Z}\frac{\partial Z}{\partial \beta}\\
    &= -\frac{h\beta\w_0}{2\pi}\cdot\left(-\frac{1}{h}\frac{2\pi}{\beta^2\w_0}\right)\\
    &= \frac{1}{\beta}.
\end{align*}The specific heat capacity $C_v$ is computed accordingly:
\begin{align*}
    C_v &= \frac{\partial \average{E}}{\partial T}\\
    &= \frac{\partial}{\partial T} \left(k_bT\right) = k_b.
\end{align*}

\section*{Task 2}
The same oscillator as in the previous task, but with a quantum mechanical point of view leads to the eigenvalues:
\begin{align*}
    E_n = \left(n + \frac{1}{2}\right)hf,
\end{align*}where $f$ is the natural frequency of the oscillator.

\subsection*{a)}
Evaluate the partition function $Z$, for a single harmonic oscillator.

\vspace*{0.5cm}\noindent
The partition function is now discrete and thus is given by:
\begin{align}
    Z &= \sum_{n=0}^\infty e^{-\beta E_n}\nonumber\\
    &= \sum_{n=0}^\infty \exp\left[-\beta\left(n + \frac{1}{2}\right)hf\right]\nonumber\\
    &=\exp\left[-\frac{\beta hf}{2}\right]\sum_{n=0}^\infty \exp\left[-\beta nhf\right]\nonumber\\
    &= \exp\left[-\frac{\beta hf}{2}\right]\frac{1}{1-\exp\left[-\beta hf\right]}.\label{eq: Parition function}
\end{align}


\subsection*{b)}
From the partition function, find the average energy $\langle E \rangle$ at a temperature $T$. Also take the limit $T\to\infty$.

\vspace*{0.5cm}\noindent
Using eq. \eqref{eq: Parition function} we can compute the average energy:
\begin{align*}
   \average{E} &= -\frac{1}{Z}\frac{\partial Z}{\partial \beta}\\
   &= -\left[\exp\left[-\frac{\beta hf}{2}\right]\frac{1}{1-\exp\left[-\beta hf\right]}\right]^{-1}\frac{\partial}{\partial \beta}\left(\exp\left[-\frac{\beta hf}{2}\right]\frac{1}{1-\exp\left[-\beta hf\right]}\right)\\
   &= \left\{\frac{\partial}{\partial \beta}\left(\exp\left[-\frac{\beta hf}{2}\right]\right)\cdot \frac{1}{1-\exp\left[-\beta hf\right]}+\frac{\partial}{\partial \beta}\left(\frac{1}{1-\exp\left[-\beta hf\right]}\right)\cdot \exp\left[-\frac{\beta hf}{2}\right]\right\}\\
   &\cdot(-1)\left[\frac{1-\exp\left[-\beta hf\right]}{\exp\left[-\frac{\beta hf}{2}\right]}\right]\\
   &= \left\{-\frac{hf}{2\left(1-\exp\left[-\beta hf\right]\right)}\exp\left[-\frac{\beta hf}{2}\right] - \frac{hf\exp\left[\beta hf\right]}{\left(\exp\left[\beta hf\right] - 1\right)^2}\cdot\exp\left[-\frac{\beta hf}{2}\right]\right\}\\
   &\cdot(-1)\left[\frac{1-\exp\left[-\beta hf\right]}{\exp\left[-\frac{\beta hf}{2}\right]}\right]\\
   &= \left\{\frac{1}{2} + \frac{\left(1-\exp\left[-\beta hf\right]\right)\cdot hf \exp\left[\beta hf\right]}{\left(\exp\left[\beta hf\right] - 1\right)^2}\right\}\\
   &= \frac{1}{2} + \frac{hf}{2}\coth\left(\frac{\beta hf}{2}\right)
\end{align*}In the limit when $T\to\infty$ one has that $\beta\to0$ which gives:
\begin{align*}
    Z_\infty &= \lim_{\beta\to0} = \left\{\frac{1}{2} + \frac{\left(1-\exp\left[-\beta hf\right]\right)\cdot hf \exp\left[\beta hf\right]}{\left(\exp\left[\beta hf\right] - 1\right)^2}\right\}\\
    &=\frac{1}{2} + 
\end{align*}


\subsection*{c)}
Compute the specific heat capacity $C_V$ for this system.

\subsection*{d)}
Find an expression for the Helmholtz free energy for this system.

\subsection*{e)}
Find an expression for the entropy of this system as a function of temperature.

\section*{Task 3}
A specific system of $N$ distinguishable particles has the Hamiltonian on the form:
\begin{align*}
    H = -\mu B\sum_{i=1}^NS_i; \quad S_i\in\{-1, 0, 1\}. 
\end{align*}

\subsection*{a)}
Write the canonical partition function for this system and hence find the free energy.  (hint:  First find the partition function for one particle!)

\newparagraph
We consider that a single particle can have three different state $S_i\in\{-1, 0, 1\}$, and thus the partition function for a single particle is given by:
\begin{align*}
    Z_1 &= \sum_{S_i\in\{-1, 0, 1\}}e^{-\beta H} = e^{\beta\mu B} + 1 + e^{-\beta\mu B}\\
    &= 2\cosh(\beta\mu B) + 1.
\end{align*}The partition function for $N$ particles is then given by:
\begin{align*}
    Z &= \prod_{i = 1}^N Z_1\\
    &= \prod_{i = 1}^N \left(2\cosh(\beta\mu B) + 1\right)\\
    &= \left(2\cosh(\beta\mu B) + 1\right)^N.
\end{align*}The free energy $\f$ is given:
\begin{align*}
    \f &= -k_bT\ln(Z)\\
    &= -Nk_bT\ln\left(2\cosh(\beta\mu B) + 1\right).
\end{align*}

\subsection*{b)}
Find the average energy $U$ and entropy $S$ and simplify these expressions in the limits $T\to0$ and $T\to\infty$.

\newparagraph
\begin{comment}
\begin{align*}
    \average{E} &= -\frac{N(2\sinh(\mu\beta B)\mu B)}{(1+\cosh(\mu\beta B))^N}\cdot (1+\cosh(\mu\beta B))^{N-1}\\
    &= -\frac{N2\sinh(\mu\beta B)\mu B}{1+\cosh(\mu\beta B)}\\
    &= -N\mu B \tanh\left(\frac{\mu\beta B}{2}\right)
\end{align*}
\end{comment}
We compute the average energy $U = \average{E}$ by the following:
\begin{align*}
    \average{E} &= -\frac{1}{Z}\frac{\partial Z}{\partial \beta}\\
    &= -\frac{N(2\sinh(\mu\beta B)\mu B)}{(1+\cosh(\mu\beta B))^N}\cdot (1+\cosh(\mu\beta B))^{N-1}\\
    &= -B\mu\frac{\sinh\left(\beta\mu B\right)}{\cosh(\beta\mu B) + 1}\\
    &= -N\mu B \tanh\left(\frac{\mu\beta B}{2}\right)
\end{align*}
The entropy $S$ is computed accordingly:
\begin{align*}
    S &= \frac{\average{E}}{T} + k_b\ln\left(Z\right)\\
    &= -BN\mu\frac{\tanh\left(\frac{\mu B}{2k_bT}\right)}{T} + k_b\ln\left(\left(2\cosh\left(\frac{\mu B}{k_bT}\right) + 1\right)^N\right)\\
    &= -BN\mu\frac{\tanh\left(\frac{\mu B}{2k_bT}\right)}{T} + Nk_b\ln\left(\left(2\cosh\left(\frac{\mu B}{k_bT}\right) + 1\right)\right).
\end{align*}
In the two limits one has the following:
\begin{align*}
    S_\infty &= \lim_{T\to\infty}\left[ -NB\mu\frac{\tanh\left(\frac{\mu B}{2k_bT}\right)}{T} + Nk_b\ln\left(\left(2\cosh\left(\frac{\mu B}{k_bT}\right) + 1\right)\right)\right]\\
    &=0,\\
    \average{E}_\infty &=\lim_{T\to\infty}\left[-NB\mu\tanh\left(\frac{\mu B}{2k_bT}\right)\right] = 0,\\
    S_0 &= \lim_{T\to0}\left[ -NB\mu\frac{\tanh\left(\frac{\mu B}{2k_bT}\right)}{T} + Nk_b\ln\left(\left(2\cosh\left(\frac{\mu B}{k_bT}\right) + 1\right)\right)\right]\\
    &= \infty,\\
    \average{E}_0 &= \lim_{T\to0}\left[-BN\mu\tanh\left(\frac{\mu B}{2k_bT}\right)\right] = \mp NB\mu.
\end{align*}The average energy is discontinuous at $T=0$, and thus has a left and right limit. The entropy has a singularity at $T=0$.

\section*{Task 4}
Consider the canonical ensemble (and the canonical partition function).

\subsection*{a)}
Show that the average energy $E = -\frac{\partial}{\partial \beta}\left(\ln(Z)\right)$. What would $E = -\frac{\partial}{\partial \beta}\left(\ln(Z_g)\right)$ be if $Z_g$ is the grand Canonical partition function?

\newparagraph
The average energy is given by:
\begin{align*}
    \average{E} &= \sum_i E_i p_i = \frac{1}{z}\sum_i E_i e^{\beta E_i}\\
    &= -\frac{1}{Z}\sum_i E_i \frac{\partial}{\partial \beta}\left(e^{-\beta E_i}\right)\\
    &= -\frac{1}{Z}\frac{\partial}{\partial \beta}\left(\sum_i E_i e^{-\beta E_i}\right)\\
    &= -\frac{\partial}{\partial \beta}\ln\left(\sum_i e^{-\beta E_i}\right)\\
    &= -\frac{\partial}{\partial \beta}\ln(Z).
\end{align*}If one instead considers the grand canonical partition function $Z_g$, and evaluates the above, one obtains:
\begin{align*}
    -\frac{\partial}{\partial \beta}\ln(Z_g) &= -\frac{\partial}{\partial \beta}\ln\left(\sum_i e^{-\beta(E_i - \mu N_i)}\right)\\
    &= \frac{\sum_i (E_i - N_i\mu)\exp\left[-\beta(E_i-N_i\mu)\right]}{\sum_i \exp\left[\beta(E_i - \mu N_i)\right]}
\end{align*}

\subsection*{b)}
Show that the fluctuations in the energy average energy:
\begin{align*}
    \Delta E^2 = \bar{E}^2 - \left(\bar{E}\right)^2
\end{align*}is proportional to the specific heat capacity. Can specific heat capacity be negative?

\newparagraph
In order to show that the fluctuations in the $\Delta E^2$ is proportional to the specific heat capacity, we first compute the fluctuations:
\begin{align*}
    \Delta E^2 &= \average{E^2} - \average{E}^2\\
    &= \frac{\partial}{\partial \beta}\left[\ln\left(Z^2\right)\right] - \left(\frac{\partial}{\partial \beta}\ln(Z)\right)^2\\
    &= \frac{\partial^2}{\partial \beta^2}\left[\ln(Z)\right].
\end{align*}The heat capacity is given by:
\begin{align*}
    C_v &= \frac{\partial \average{E}}{\partial T} \\
    &= \frac{\partial}{\partial T}\frac{\partial}{\partial \beta}\left[\ln(Z)\right]\\
    &=\frac{1}{k_b T^2}\frac{\partial^2}{\partial \beta^2}\left[\ln(Z)\right].
\end{align*}By this, one has that $\Delta E^2 = k_b T^2 C_v$, i.e. $C_v \propto \Delta E^2$. Moreover, the specific heat capacity can be negative. This often occurs in stellar system with the phenomena called 'negative heat capacity', as well as in ions.
In terms of the expression for the specific heat, the following can be debated:
\begin{align*}
    C_v &= \underbrace{\frac{1}{k_bT^2}}_{>= 0}\underbrace{\Delta E^2}_{?}.
\end{align*}The first term is always zero or positive, however zero temperature have never been observed, however the second term could potentially be negative. This would then in turn lead to a negative heat capacity.


\section*{Task 5}
Consider the following definition of entropy,
\begin{align*}
    S = -k_b\sum_s P(s)\ln(P(s)),
\end{align*}where $s$ is a microstate of the system and $P(s)$ is the probability of the system being in the microstate $s$.

\subsection*{a)}
Show that this definition of entropy is equivalent to our usual definitions, for a system in thermal equilibrium with a reservoir at temperature $T$.

\newparagraph
One can rewrite the following expression as:
\begin{align*}
    S &= -k_b\sum_s \ln\left[P(s)^{P(s)}\right]
\end{align*}

\subsection*{b)}
Show that this definition is consistent even for a system in contact with a reservoir at fixed temperature $T$ and fixed chemical potential $\mu$. (Hint: remember $kT\ln(Z_g)=PV$ where $Z_g$ is the grand partition function!).


\end{document}
 
