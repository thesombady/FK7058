\documentclass[a4paper]{article}

% --- Packages ---

\usepackage{a4wide}
\usepackage[utf8]{inputenc}
\usepackage{amsmath}
\usepackage{mathtools}
\usepackage{amssymb}
\usepackage[english]{babel}
\usepackage{mdframed}
\usepackage{systeme,}
\usepackage{lipsum}
\usepackage{relsize}
\usepackage{caption}
\usepackage{tikz}
\usepackage{tikz-3dplot}
\usetikzlibrary{shapes.geometric}
\usepackage{pgfplots}
\usepackage{pgfplotstable}
\pgfplotsset{compat=newest}%1.7}
\usepackage{harpoon}%
\usepackage{graphicx}
\usepackage{wrapfig}
\usepackage{subcaption}
\usepackage{authblk}
\usepackage{float}
\usepackage{listings}
\usepackage{xcolor}
\usepackage{chngcntr}
\usepackage{amsthm}
\usepackage{comment}
\usepackage{commath}
\usepackage{hyperref}%Might remove, adds link to each reference
\usepackage{url}
\usepackage{calligra}
\usepackage{pgf}

% --- Bibtex ---

%\usepackage[backend = biblar,]{bibtex}

%\addbibliografy(ref.bib)

% --- Commands --- 

\newcommand{\w}{\omega}
\newcommand{\trace}{\text{Tr}}
\newcommand{\grad}{\mathbf{\nabla}}
%\newcommand{\crr}{\mathfrak{r}}
\newcommand{\laplace}{\nabla^2}
\newcommand{\newparagraph}{\vspace{.5cm}\noindent}

% --- Math character commands ---

\newcommand{\curl}[1]{\mathbf{\nabla}\times \mathbf{#1}}
\newcommand{\dive}[1]{\mathbf{\nabla}\cdot \mathbf{#1}}
\newcommand{\res}[2]{\text{Res}(#1,#2)}
\newcommand{\fpartial}[2]{\frac{\partial #1}{\partial #2}}
\newcommand{\rot}[3]{\begin{vmatrix}\hat{x}&\hat{y}&\hat{z}\\\partial_x&\partial_y&\partial_z\\#1&#2&#3 \end{vmatrix}}
\newcommand{\average}[1]{\langle #1 \rangle}
\newcommand{\ket}[1]{|#1\rangle}
\newcommand{\bra}[1]{\langle #1|}


%  --- Special character commands ---

\DeclareMathAlphabet{\mathcalligra}{T1}{calligra}{m}{n}
\DeclareFontShape{T1}{calligra}{m}{n}{<->s*[2.2]callig15}{}
\newcommand{\crr}{\mathcalligra{r}\,}
\newcommand{\boldscriptr}{\pmb{\mathcalligra{r}}\,}


\title{Handin 5 :  FK7058}
\author{Author : Andreas Evensen}
\date{Date: \today}

% --- Code ---

\definecolor{codegreen}{rgb}{0,0.6,0}
\definecolor{codegray}{rgb}{0.5,0.5,0.5}
\definecolor{codepurple}{rgb}{0.58,0,0.82}
\definecolor{backcolour}{rgb}{0.95,0.95,0.92}

\lstdefinestyle{mystyle}{
    backgroundcolor=\color{backcolour},   
    commentstyle=\color{codegreen},
    keywordstyle=\color{magenta},
    numberstyle=\tiny\color{codegray},
    stringstyle=\color{codepurple},
    basicstyle=\ttfamily\footnotesize,
    breakatwhitespace=false,         
    breaklines=true,                 
    captionpos=b,                    
    keepspaces=true,                 
    numbers=left,                    
    numbersep=5pt,                  
    showspaces=false,                
    showstringspaces=false,
    showtabs=false,                  
    tabsize=2
}

\lstset{style=mystyle}

\begin{document}

\maketitle

\begin{align*}
    \mathcal{H} = -J \sum_{\average{i,j}}S_iS_j - \Delta \sum_{i=1}^NS_i^2 - H \sum_{i = 1}^NS_i
\end{align*}

\subsection*{Problem 1}
Make the usual mean-field approximation to convert the term with nearest-neighbour interactions to a term over single spins and write the mean-field Hamiltonian.

\newparagraph
\textbf{Answer:} We rewrite the above Hamiltonian in the following manner:
\begin{align*}
    \mathcal{H} &= -J \sum_{\average{i,j}}S_iS_j - \Delta \sum_{i=1}^NS_i^2 - H \sum_{i = 1}^NS_i\\
    &= -\sum_{\average{i,j}}J_{i,j}\average{S_j}S_i - \Delta \sum_{i=1}^NS_i^2 - H \sum_{i = 1}^NS_i\\
    &= -\sum_i (H + \sum_{j}\average{S_j})S_i - \Delta S_i^2\\
    &= -\sum_i S_i(h_i + \Delta S_i),
\end{align*}where $h_i = H - J\sum_j^{2d} \average{S_j}$.


\subsection*{Problem 2}
Determine a self-consistent equation for the magnetization $M$.

\newparagraph
\textbf{Answer:} In order to find the self-consistent equation for the magnetization, one has to find the partition function:
\begin{align*}
    Z &= \sum_{\{S_j\}}e^{-\beta\mathcal{H}}= \sum_{\{S_j\}} e^{\beta\sum_i S_i(h_i + \Delta S_i)}\\
    &= \sum_{\{S_j\}} \prod_{i = 1}^N \exp\left[\beta S_i(h_i + \Delta S_i)\right]=  \prod_{i = 1}^N \sum_{\{S_j\}}\exp\left[\beta S_i(h_i + \Delta S_i)\right]\\
    &= \prod_{i = 1}^N \left(\exp\left[\beta( h_i + \Delta)\right]+\exp\left[-\beta h_i + \beta\Delta\right] + 1\right)\\
    &= \prod_{i = 1}^N 2e^{\beta \Delta }\left(\cosh(\beta h_i) + \frac{e^{-\beta \Delta}}{2}\right)= \left[2e^{\beta \Delta}\left(\cosh(\beta (H + 2dJM))+\frac{e^{-\beta \Delta}}{2}\right)\right]^N.
\end{align*}From this, one can compute the free energy as:
\begin{align*}
    \mathcal{F} &= -k_bT\ln(Z)\\
    &= -k_bTN\ln\left[2e^{\beta \Delta}\left(\cosh(\beta (H + 2dJM))+\frac{e^{-\beta \Delta}}{2}\right)\right].
\end{align*}The magnetization is thus given by:
\begin{align*}
    M &= - \frac{1}{N}\frac{\partial \mathcal{F}}{\partial H}\\
    &= \frac{\sinh(\beta (H + 2dJM))}{\cosh(\beta (H + 2dJM)) + \frac{e^{-\beta \Delta}}{2}}.
\end{align*}


\subsection*{Problem 3}
In all that follows we put $H = 0$. From the above equation obtain an equation for the critical temperature at which the Blume-Chapel models undergoes a continuous phase transition.
(Hint: Keep only the highest order term in both the numerator and denominator of the self-consistent equation! You will get a transcendental equation for the critical temperature.)

\newparagraph
\textbf{Answer:} If $H = 0$, we get the following magnetization:
\begin{align*}
    M&= \frac{\sinh(\beta 2dJM)}{\cosh(\beta 2dJM) + \frac{e^{-\beta \Delta}}{2}} = f(M, \beta)\\
\end{align*}Differentiating with respect to $M$ gives:
\begin{align}
    \fpartial{f}{M} &= \frac{\beta 2 d J\cosh\left(\beta 2 d J M \right)}{\cosh\left(\beta 2 d J M\right) + \frac{e^{-\beta \Delta}}{2}} - \frac{\beta 2 d J\sinh^2\left(\beta 2 d J M\right)}{\left(\cosh\left(\beta 2 d J M\right) + \frac{e^{-\beta \Delta}}{2}\right)^2}\nonumber\\
    \fpartial{f}{M}\Big|_{M= 0} &= \frac{\beta 2 d J}{1 + \frac{e^{-\beta \Delta}}{2}} = 1\label{eq: Critical temperature derivation}.
\end{align}Letting $\beta = \frac{1}{k_bT_c}$, by solving \eqref{eq: Critical temperature derivation} for $T_c$, we get the critical temperature:
\begin{align*}
    k_bT_c &= \frac{2d J}{1 + \frac{e^{-\frac{\Delta}{k_bT_c}}}{2}}.
\end{align*}


\subsection*{Problem 4}
Taylor-expand the self-consistent equation keeping terms up to $M^5$ and from this construct a Landau free energy.

\newparagraph
\textbf{Answer:} Using the self-consistent equation, with $H = 0$, we rewrite the equation as follows:
\begin{align*}
    f(M) = M\cdot\left(\cosh\left(\frac{2dJM}{k_bT}\right) + \frac{e^{-\frac{\Delta}{k_bT}}}{2}\right) = \sinh\left(\frac{2dJM}{k_bT}\right) = Q(m).
\end{align*}From this, we Taylor-expand each side of the equation: with five terms:
\begin{align*}
    \text{LHS:}\quad & f(0) + f'(0)\cdot M + \frac{f''(0)}{2}M^2 + \frac{f^{(3)}(0)}{6}M^3 + \frac{f^{(4)}(0)}{24}M^4 + \frac{f^{(5)}(0)}{120}M^5\\
    &= 0 + \left(1 + \frac{1}{2}e^{-\frac{\Delta}{k_bT}}\right)\cdot M + (0)\cdot M^2 + 2\left(\frac{Jd}{k_bT}\right)^2\cdot M^3 + (0)\cdot M^4 + \frac{2}{3}\left(\frac{Jd}{k_bT}\right)^4\cdot M^5\\
    \text{RHS:}\quad & Q(0) + Q'(0)\cdot M + \frac{Q''(0)}{2}M^2 + \frac{Q^{(3)}(0)}{6}M^3 + \frac{Q^{(4)}(0)}{24}M^4 + \frac{Q^{(5)}(0)}{120}M^5\\
    &= 0 + 2\left(\frac{Jd}{k_bT}\right)\cdot M + \left(0\right)\cdot M^2 + \frac{8}{6}\left(\frac{Jd}{k_bT}\right)^3\cdot M^3 + (0)\cdot M^4 + \frac{4}{15}\left(\frac{Jd}{k_bT}\right)^5\cdot M^5.
\end{align*}Moving the RHS to the LHS, and then doing the following:
\begin{align*}
    0 &= -\left(2\frac{Jd}{k_bT} - 1 - \frac{1}{2}e^{-\frac{\Delta}{k_bT}}\right)\cdot M - \left(\frac{8}{6}\left(\frac{Jd}{k_bT}\right)^3 - 2\left(\frac{Jd}{k_bT}\right)^2\right)\cdot M^3\\
    &-\left(\frac{4}{15}\left(\frac{Jd}{k_bT}\right)^5 - \frac{2}{3}\left(\frac{Jd}{k_bT}\right)^4\right)\cdot M^5
\end{align*}Matching the terms with $a_i$ we get the following:
\begin{align*}
    \fpartial{\mathcal{L}}{M} &= a_1M + a_3M^3 + a_5M^5,\\
    \implies \mathcal{L} &= \frac{a_1}{2}M^2 + \frac{a_3}{4}M^4 + \frac{a_5}{6}M^6.
\end{align*}


\subsection*{Problem 5}
At the critical temperature you found in a previous step, show that the coefficient of the quadratic term goes to zero, signalling a continuous phase transition.

\newparagraph
\textbf{Answer:} The quadratic term:
\begin{align*}
    &\left(1 + \frac{1}{2}e^{-\frac{\Delta}{k_bT}} - 2\frac{Jd}{k_bT}\right),\\
    \implies &\left(1 + \frac{1}{2}e^{-\frac{\Delta}{k_bT}}\right) = 2\frac{Jd}{k_bT}\\
\end{align*}goes towards zero when entering a phase-transition when the temperature is at the critical temperature. 
\begin{align*}
    \left(1 + \frac{1}{2}e^{-\frac{\Delta}{k_bT_c}}\right) &= 2\frac{Jd}{k_bT_c}\\
    \left(1 + \frac{1}{2}e^{-\frac{\Delta}{k_bT_c}}\right) &= 2Jd\cdot\left(\frac{1 + \frac{e^{-\frac{\Delta}{k_bT}}}{2}}{2dJ}\right)\\
    \left(1 + \frac{1}{2}e^{-\frac{\Delta}{k_bT_c}}\right) &= \left(1 + \frac{e^{-\frac{\Delta}{k_bT}}}{2}\right).
\end{align*}They are equivilant, and thus the coefficient of the quadratic term goes to zero at the critical temperature.

\subsection*{Problem 6}
Show that there is also another temperature and a value of $\Delta$, here the coefficients of both the quadratic and quartic terms go to zero.
Hence, having another parameter in the Hamiltonian (in this case $\Delta$) can lead to a situation where we need to the sixth-order term in the Landau free energy.
We will see the implication of this in the next handin!

\newparagraph
\textbf{Answer:}

\end{document}
 
